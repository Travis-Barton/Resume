%%%%%%%%%%%%%%%%%
% This is an sample CV template created using altacv.cls
% (v1.1.4, 27 July 2018) written by LianTze Lim (liantze@gmail.com). Now compiles with pdfLaTeX, XeLaTeX and LuaLaTeX.
% 
%% It may be distributed and/or modified under the
%% conditions of the LaTeX Project Public License, either version 1.3
%% of this license or (at your option) any later version.
%% The latest version of this license is in
%%    http://www.latex-project.org/lppl.txt
%% and version 1.3 or later is part of all distributions of LaTeX
%% version 2003/12/01 or later.
%%%%%%%%%%%%%%%%

%% If you need to pass whatever options to xcolor
\PassOptionsToPackage{dvipsnames}{xcolor}

%% If you are using \orcid or academicons
%% icons, make sure you have the academicons 
%% option here, and compile with XeLaTeX
%% or LuaLaTeX.
% \documentclass[10pt,a4paper,academicons]{altacv}

%% Use the "normalphoto" option if you want a normal photo instead of cropped to a circle
% \documentclass[10pt,a4paper,normalphoto]{altacv}

\documentclass[10pt,a4paper]{altacv}
%% AltaCV uses the fontawesome and academicon fonts
%% and packages. 
%% See texdoc.net/pkg/fontawecome and http://texdoc.net/pkg/academicons for full list of symbols.
%% 
%% Compile with LuaLaTeX for best results. If you
%% want to use XeLaTeX, you may need to install
%% Academicons.ttf in your operating system's font 
%% folder.


% Change the page layout if you need to
\geometry{left=1cm,right=9cm,marginparwidth=6.8cm,marginparsep=1.2cm,top=1.25cm,bottom=1.25cm,footskip=2\baselineskip}

% Change the font if you want to.

% If using pdflatex:
\usepackage[T1]{fontenc}
\usepackage[utf8]{inputenc}
\usepackage[default]{lato}

% If using xelatex or lualatex:
% \setmainfont{Lato}

% Change the colours if you want to
\definecolor{Brown}{HTML}{D2691E}
\definecolor{SlateGrey}{HTML}{2E2E2E}
\definecolor{LightGrey}{HTML}{666666}
\colorlet{heading}{Sepia}
\colorlet{accent}{Brown}
\colorlet{emphasis}{SlateGrey}
\colorlet{body}{LightGrey}

% Change the bullets for itemize and rating marker
% for \cvskill if you want to
\renewcommand{\itemmarker}{{\small\textbullet}}
\renewcommand{\ratingmarker}{\faCircle}
%% sample.bib contains your publications
\addbibresource{sample.bib}

\usepackage[colorlinks]{hyperref}

\begin{document}

\name{Travis Barton}
\tagline{Statistical Data Scientist/Classifier}
\photo{2.8cm}{Me}
\personalinfo{%
  % Not all of these are required!
  % You can add your own with \printinfo{symbol}{detail}
  \email{Travis.Barton@sjsu.edu}
  \phone{408-826-2999}
  \mailaddress{403 brown st, San Jose ca}
  \location{San Jose, ca}
  \homepage{wbbpredictions.com/about-us/}
  \linkedin{linkedin.com/in/travis-barton-269613117/}
  \github{github.com/Travis-Barton}
  %% You MUST add the academicons option to \documentclass, then compile with LuaLaTeX or XeLaTeX, if you want to use \orcid or other academicons commands.
%   \orcid{orcid.org/0000-0000-0000-0000}
}

%% Make the header extend all the way to the right, if you want. 
\begin{fullwidth}
\makecvheader
\end{fullwidth}

%% Depending on your tastes, you may want to make fonts of itemize environments slightly smaller
% \AtBeginEnvironment{itemize}{\small}


%% Provide the file name containing the sidebar contents as an optional parameter to \cvsection.
%% You can always just use \marginpar{...} if you do
%% not need to align the top of the contents to any
%% \cvsection title in the "main" bar.
\cvsection[page1sidebar]{Experience}

\cvevent{Lecturer}{San Jose State University}{August 2017 -- Ongoing}{San Jose, ca}
\begin{itemize}
\item Lead lectures for classes larger than 60.  
\item Fostered a safe and independent learning environment.
\end{itemize}

%\divider
%
%\cvevent{Data Analyst \tiny{unpaid}}{United Way Monterey}{January 2016 -- August 2016}{Monterey, ca}
%\begin{itemize}
%\item prepared/analyzed call data for marketing presentations
%
%\item Provided buisness insight through the interpretation of figures and model parameters
%\end{itemize}

\cvsection{Projects}

\cvevent{Digit Recognition and Titanic survival: \\ An introduction to machine learning \tiny{team leader}}{wbbpredictions}{4 months}{}
\begin{itemize}
\item Created guides and learning materials for new data scientists to use in order to introduce themselves into the world of machine learning. 
\item Topics included: SVM, KNN, Random Forest, PCA, MCA, and Logistic Regression.
\end{itemize}

\divider

\cvevent{Mass Labeling Tool \tiny{team member}}{Intuit + San Jose State}{5 months/ongoing}{}
\begin{itemize}
    \item Created a python based MLT framework for discerning customer comment intent and classifying their comment accordingly. 
    \item Built a custom UI for experts to use to interact with our model.
\end{itemize}

\divider
\cvevent{SubReddit Auto Tagging Bot}{Personal}{ongoing}{}
The use of Natural Language Processing to determine intent is a task that has recently had phenomenal breakthroughs. The use of multi-modeled neural networks (such as in the Universal Language Model) have allowed data scientist to use fractions of the amount of labeled data to produce the same state of the art results as traditional methods. The SubReddit auto tagger (SAT) will be a bot that is able to analyze, cluster and tag posts from any text/link based SubReddit autonomously, using these breakthrough techniques. 

\medskip

\cvsection{A Day of My Life}

% Adapted from @Jake's answer from http://tex.stackexchange.com/a/82729/226
% \wheelchart{outer radius}{inner radius}{
% comma-separated list of value/text width/color/detail}
\wheelchart{1.5cm}{0.5cm}{%
  5/8em/accent!30/{Sleeping\\\textit{(beautiful sleep)}}, 
  3/7em/accent!40/Pursuing personal projects,
  8/8em/accent!60/Studying + programming,
  2/10em/accent/{Playing ultimate, rock climbing or relaxing},
  4/6em/accent!20/Preparing lectures and club materials
}

\clearpage
\cvsection[page2sidebar]{Publications}

\nocite{*}

\printbibliography[heading=pubtype,title={\printinfo{\faBook}{Books}},type=book]

\divider

\printbibliography[heading=pubtype,title={\printinfo{\faFileTextO}{Journal Articles}},type=article]

\divider

\printbibliography[heading=pubtype,title={\printinfo{\faGroup}{Conference Proceedings}},type=inproceedings]

%% If the NEXT page doesn't start with a \cvsection but you'd
%% still like to add a sidebar, then use this command on THIS
%% page to add it. The optional argument lets you pull up the 
%% sidebar a bit so that it looks aligned with the top of the
%% main column.
% \addnextpagesidebar[-1ex]{page3sidebar}

\end{document}
